\problemname{Slide Count}

In your programming class, you are given an assignment to analyze an
integer array using a sliding window algorithm.  Specifically, given
$N$ integers $w_1, \ldots, w_N$ and some constant $C$, the sliding
window algorithm maintains start and end indices $s$ and $e$ such that
\begin{itemize}
\item initially $s = e  = 1$;
\item as long as $s \leq N$:
  \begin{itemize}
  \item if $e+1 > N$, then increment $s$;
  \item else if $w_s + \cdots + w_{e+1} > C$, then
    increment $s$;
  \item else increment $e$.
  \end{itemize}
\end{itemize}
During the execution of this algorithm, each distinct pair of indices
$(s,e)$ defines a window.  An element $w_i$ belongs to the window
defined by $(s,e)$ if $s \leq i \leq e$.  Notice that if $s > e$, the
window is empty.

Consider the first sample input below.  The windows appearing
during the execution of the algorithm are defined by $(1,1)$, $(1,2)$,
$(1,3)$, $(2,3)$, $(3,3)$, $(3,4)$, $(4,4)$, $(5,4)$, $(5,5)$, and
$(6,5)$.

For each element $w_i$, determine how many different windows it
belongs to during the execution of the sliding window algorithm.

\section*{Input}

The first line of input contains two integers $N$
($1 \leq N \leq 100\,000$), which is the number of elements, and $C$
($1 \leq C \leq 1\,000\,000$), which is the sliding window constant. 

The next line contains $N$ integers $w_1, \ldots, w_N$ 
($0 \leq w_i \leq C$).

\section*{Output}
For each element, in order, display the number of different
windows it belongs to during the execution of the algorithm.

%%% Local Variables:
%%% mode: latex
%%% TeX-master: t
%%% End:
