\problemname{Antialiasing}

\begin{center}
\includegraphics{diagram.png}
\end{center}

To reduce aliasing effects, computer graphics systems render a polygon 
by setting the brightness of each pixel proportional to the area of the 
pixel inside the polygon.  If a pixel is completely inside the polygon, 
that pixel is set to the brightest intensity.  If only half of the pixel 
is inside the polygon, then the pixel is set halfway between darkest and 
brightest.

Given a convex polygon and a pixel location, determine the fraction of
the pixel's area that is inside the polygon.  Each pixel is square,
and is indexed by its row and column coordinates $r$ and $c$.  If a
vertex of the polygon is located at $(r,c)$, then the vertex is
located at the center of the pixel at row $r$ and column $c$.
Rows are numbered from $0$ starting from the top row, and columns are numbered
from $0$ starting from the leftmost column.


\section*{Input}

The first line of input specifies two integers $N$
($3 \leq N \leq 100$), which is the number of vertices in the convex
polygon, and $Q$ ($1 \leq Q \leq 1\,000$), which is the number of
queries.  The next $N$ lines each contains two integers $r$ and $c$
giving the coordinates of the polygon in
counterclockwise order.  The next $Q$ lines each contains two integers
$r$ and $c$ indicating the coordinates of
the pixel we are interested in.  It is guaranteed that the area of the
polygon is positive.  All coordinates satisfy $0 \leq r \leq 1\,000$
and $0 \leq c \leq 1\,000$.

\section*{Output}

For each query, display a line indicating the fraction of the pixel's
area inside the polygon.  The fraction should be in
lowest terms.  

%%% Local Variables:
%%% mode: latex
%%% TeX-master: t
%%% End:
